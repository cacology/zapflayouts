\documentclass{memoir}
\usepackage{mathpazo}
\usepackage{xcolor}
\usepackage[a4paper,landscape]{geometry}
\usepackage{dingbat}

\begin{document}
\noindent \normalsize \rightpointright
\tiny \textcolor{red}{\MakeUppercase{D. Berkeley
    Updike:}} In every period there have been better of worse types
employed in better or worse ways. The
better types \\
\footnotesize employed in better ways have been used by the educated printer
acquainted with standards and history, directed by taste and a
sense \\
\small of fitness of things, and facing the industrial conditions
and
the needs of his time. Such men have made of printing \\
\normalsize an art. The poorer types and methods have been employed by
printers
ignorant of standards and caring \\
\large alone for commercial success. To these, printing has been
simply a
trade. The typography \\
\Large of a nation has been good or bad, as one or other of these
classes had
the supre-\\
\LARGE macy. And to-day any intelligent printer can educate his
taste, so to \\
\huge choose types for his work \& so to use them, that he will \\
\Huge help printing to be an art rather than a
trade. \\
\textcolor{red}{ABCDEFGHIJKLMNOPQRSTUVWXYZ}
\end{document}
