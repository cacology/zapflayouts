\documentclass{memoir}
\usepackage{mathpazo}
\usepackage{xcolor}
\usepackage[a4paper,landscape]{geometry}
\usepackage{dingbat}
\usepackage{anyfontsize}

\begin{document}
\setlength{\parfillskip}{0pt}
{\noindent \rightpointright \fontsize{6}{7}\selectfont
  \textcolor{red}{\MakeUppercase{D. Berkeley Updike:}} In every period
  there have been better of worse types employed in better or worse
  ways. The better types}

{\noindent\fontsize{8}{9}\selectfont employed in better ways have been used
  by the educated printer acquainted with standards and history,
  directed by taste and a sense}

{\noindent \small of fitness of things, and facing the
  industrial conditions and the needs of his time. Such men have made
  of printing}

{\noindent \normalsize an art. The poorer types and methods
  have been employed by printers ignorant of standards and caring}

{\noindent \fontsize{11}{16}\selectfont alone for commercial
  success. To these, printing has been simply a trade. The typography
  }

{\noindent \fontsize{12}{18}\selectfont of a nation has been good or bad, as one or
  other of these classes had the supre-}

{\noindent \LARGE macy. And to-day any intelligent printer can educate his
taste, so to}

{\noindent \huge choose types for his work \& so to use them, that he will}

{\noindent \Huge help printing to be an art rather than a
trade.}

{\noindent \Huge \textcolor{red}{ABCDEFGHIJKLMNOPQRSTUVWXYZ}}
\end{document}
